%!TEX root = ../Thesis.tex
\chapter{Summary}
To develop a successful oral formulation of Insulin for treatment of type-2 diabetes patients would be great mile stone in terms of convenience. Besides protecting insulin from enzymatic cleavage in the small intestine, the intestinal epithelia is a significant barrier to overcome. Absorption enhancers are needed to ensure even a few percent of insulin are taken up. The major class of absorption enhancers is surfactant-like enhancers and is thought to promote absorption by mildly perturbing the epithelial membranes of the small intestine. The Caco-2 (Carcinoma Colon) cells can grow an artificial epithelial layer, and are used to test the potency of new absorption enhancers. In Caco-2 cells all reagents are pre-dissolved, and the assay cannot therefore predict critical solubility issues in the final tablet formulation. This project was aimed to identify new absorption enhancers, that are both potent and sufficiently soluble. Quantitative structural activity relationship (QSAR) modeling is an empiric approach to learn relationships between molecular formulas and the biochemical properties using statistical models. Molecular formulas are first encoded with different descriptor algorithms and translated into molecular descriptors. Some simple molecular descriptors simply count types of molecules or functional groups, more advanced algorithms simulate the 3D structure of the molecule to e.g. predict the dipole moment.
A public data set disclosing the potency of 42 absorption enhancers in Caco-2 was used to build a QSAR model to screen for new potent compounds. A proof-of-concept and likely the first QSAR model to predict absorption enhancement has been published. Likewise was another QSAR model built to predict solubility of absorption enhancers. After initial the proof-of-concept modeling, focus was shifted towards more academic aspects of the project. Supervised regression was used to learn relationships between molecular descriptors and potency or solubility. The random forest algorithm was found superior to multiple linear regression in terms of cross validated prediction accuracy. However, unlike multiple linear regression, an explicitly stated random forest model is complex, and therefore difficult to interpret and communicate. Any supervised regression model can be understood as a high dimensional surface connecting any possible combination of molecular properties with a given prediction. For random forests, it was discovered that a method, feature contributions, was especially useful to decompose model structure into isolated main effects and interactions effects that could be visualized and understood more easily. A novel method forest floor and R package forestFloor was developed to improve interpretation of random forest models. Better interpretation of random forest models is an exciting interdisciplinary field, as it allows investigators of many background to find fairly complicated relationships in data sets without in advance specifying what parameters to estimate. Forest floor was used to explain how potency were predicted by the random forest model.